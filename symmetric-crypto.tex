
\section{Symmetric Cryptography}

\paragraph{One-time pad}
Plaintext $p$, key $k$ such that $|p| = |k|$.
Ciphertext $c = p \xor k$.

If $k$ u.a.r. and only used once then the OTP is \textbf{perfectly secure}, i.e.
$ \Pr[P=p|C=c] = \Pr[P=p]$.

Note: keys can re-occur (as a result of random sampling) but they must not be re-used (i.e. the adversary must not be aware that the same key is used).

Issues: same lengths, key distribution, single use.

\subsection{Block Ciphers}

\paragraph{Block cipher}
A block cipher with key length $k$ and block size $n$ consists of two efficiently computable permutations\footnote{Encipher and decipher}:
$$ E: \setzeroone^k \times \setzeroone^n \mapsto \setzeroone^n \quad
D: \setzeroone^k \times \setzeroone^n \mapsto \setzeroone^n $$
such that for all keys $K$ $D_K$ is the inverse of $E_K$
(where we write $E_K$ short for $E(K, \cdot)$).

\paragraph{Security notions}
Known plaintext attack, chosen plaintext attack, chosen ciphertext attack.
Exhaustive key search on $(P,C)$ pairs -- no attack should be better, else we throw the cipher away.

\paragraph{Pseudo-randomness}
\begin{itemize}
\item Adversary $\A$ interacts either with block cipher $(E_K, D_K)$ or a truly random permutation $(\Pi, \Pi^{-1})$.
\item A block cipher is called a \textbf{pseudo-random permutation PRP} if no efficient\footnote{Quantified by runtime + number of oracle queries.} $\A$ can tell the difference between $E_K$ and $\Pi$ (no access to the inverse).
\item A block cipher is called a \textbf{strong-PRP} if no efficient $\A$ can tell the difference between $(E_K, D_K)$ and $(\Pi, \Pi^{-1})$.
\end{itemize}

\begin{figure}[h]
    \centering
	\includegraphics[scale=0.4]{images/prp.png}
    \caption{PRP game}
    \label{fig:prp}
\end{figure}

The advantage is defined as:
$$
\mathbf{Adv}^{PRP}_E (\A)
= 2 \cdot \Big| \Pr[ \text{Game }\textbf{PRP}(\A, E) \Rightarrow \text{true} ] - \frac{1}{2} \Big|
$$
where the probability is over the randomness of $b, K, \Pi, \A$.
Note that:
$$
\Pr[ \text{Game }\textbf{PRP}(\A, E) \Rightarrow \text{true} ]
= \Pr[b'=b]
$$
Also see the Advantage Rewriting Lemma (\ref{adv-rewr-lemm}).

\paragraph{Constructing block ciphers}
In general: keyed round function that is repeated many times.

\begin{itemize}
\item Feistel cipher: halved blocks crossing back and forth, e.g. DES
\item Substitution-permutation network: confusion + diffusion, e.g. AES
\end{itemize}

\paragraph{Electronic Code Book (ECB) mode}
Same plaintext always maps to the same ciphertext (deterministic).
Thus serious leakage, don't use.

\begin{figure}[h]
    \centering
	\includegraphics[scale=0.4]{images/ecb.png}
    \caption{ECB mode}
    \label{fig:ecb}
\end{figure}

\paragraph{Cipher Block Chaining (CBC) mode}
Use u.a.r. IV/previous ciphertext block to randomise encryption.

A bit flip in $C_i$ completely scrambles/randomises $P_i$ and flips the same bit in $P_{i+1}$.

Caveats: non-random IV, padding oracle attack, ciphertext block collisions (after using the same key for $2^{n/2}$ blocks by the birthday bound).

\begin{figure}[h]
    \centering
	\includegraphics[scale=0.4]{images/cbc-enc.png}
	\includegraphics[scale=0.4]{images/cbc-dec.png}
    \caption{CBC mode (left: encipher, right: decipher)}
    \label{fig:cbc}
\end{figure}

\paragraph{Counter (CTR) mode}
Incrementing counter is encrypted with block cipher to produce a pseudo-random value to xor the plaintext block with.

Effectively a stream cipher producing OTP keys.
$E_K$ does not even need to be invertible.
No padding needed, can just truncate the last block.
A bit flip it $C_i$ flips the same bit in $P_i$.

Caveats: counter must not repeat/wrap around (else xor of ciphertexts = xor of plaintexts).

\begin{figure}[h]
    \centering
	\includegraphics[scale=0.4]{images/ctr.png}
    \caption{CTR mode}
    \label{fig:ctr}
\end{figure}



\subsection{Symmetric Encryption}

\paragraph{Symmetric Encryption Scheme}
is a triple $ SE = (KGen, Enc, Dec) $.
We have key space $\K = \setzeroone^k$, message space $\M = \setzeroone^*$\footnote{In reality we might have a maximum plaintext length.} and ciphertext space $\C = \setzeroone^*$.
For correctness, we have $ Dec_K(Enc_K(m))=m $.

\paragraph{IND-CPA Security}
Informally:
computational version of perfect security --
an efficient adversary cannot compute anything useful from a ciphertext (e.g. hide every bit of the plaintext).
Equivalent to \emph{semantic security}.

Formally:
For any efficient adversary $\A$, given the encryption of one of two equal-length messages of its choice,
$\A$ is unable to distinguish which one of the two messages was encrypted.

In the security game, $\A$ gets access to a \emph{Left-or-Right encryption oracle}.
The advantage of $\A$ is:
$$
\adv^{IND-CPA}_{SE} (\A)
= 2 \cdot \Big| \Pr[ \text{Game }\textbf{IND-CPA}(\A, SE) \Rightarrow \text{true} ] - \frac{1}{2} \Big|
$$

Notes:
Deterministic schemes \textbf{cannot} be IND-CPA secure (why?).
CBC and CTR mode (if used properly) can be proven to be IND-CPA secure (assuming that $Enc$ is a PRP-secure block cipher).

Caveats:
No integrity. Says nothing about messages of non-equal length. No chosen ciphertexts.

\begin{figure}[h]
    \centering
	\includegraphics[scale=0.4]{images/ind-cpa.png}
    \caption{IND-CPA game}
    \label{fig:ind-cpa}
\end{figure}

\paragraph{Advantage Rewriting Lemma}\label{adv-rewr-lemm}
Let $b$ be a uniformly random bit and $b'$ the output of some algorithm. Then:
\begin{align*}
2 \Big| \Pr[b'=b] - \frac{1}{2} \Big|
&= \Big| \Pr[b'=1|b=1] - \Pr[b'=1|b=0] \Big| \\
&= \Big| \Pr[b'=0|b=0] - \Pr[b'=0|b=1] \Big|
\end{align*}

\paragraph{Difference Lemma}
Let $Z, W_1, W_2$ be events. If
$$
(W_1 \wedge \neg Z) \text{ occurs if and only if } (W_2 \wedge \neg Z) \text{ occurs}
$$
then
$$
\Big| \Pr[W_2] - \Pr[W_1] \Big| \leq \Pr[Z]
$$
In practice: $Z$ is a bad event that rarely happens, $W_1, W_2$ are when $\A$ wins in security games $G_1, G_2$.
Useful for \emph{game hopping} proofs.

\paragraph{PRP-PRF Switching Lemma}
Let $E$ be a block cipher.
Then for any algorithm $\A$ making $q$ queries:
$$
\Big| \adv^{PRP}_{E} (\A) - \adv^{PRF}_{E} (\A) \Big| \leq \frac{q^2}{2^{n+1}}
$$



\subsection{Attacks}

\paragraph{Padding}
Added before encryption to expand plaintext to a multiple of the block size, i.e. \\$pad(\cdot): \setzeroone^* \mapsto \{ \setzeroone^n \}^*$.
Must be expanding (why?) and efficiently computable.
May be either randomised or deterministic.
\\
E.g. TLS padding: appends $t+1$ bytes of value $t$.

Problem: adversary can detect padding errors (explicit error messages, logs, timing differences).

\paragraph{Padding Oracle}
Given a ciphertext $C$, returns whether the padding of the decryption is valid or invalid.
Leaks 1 bit of information.
Kind of a chosen ciphertext attack, thus \underline{not} covered by IND-CPA security!
Main problem: no ciphertext integrity.

In practice:
$\A$ needs 128 oracle queries on average to recover one plaintext byte.
Repeat for all bytes and all blocks for full plaintext recovery.
Additional practical details to consider.
For TLS attacks, see Lucky 13 and POODLE.

\begin{figure}[h]
    \centering
	\includegraphics[scale=0.6]{images/cbc-padding-oracle-attack.png}
    \caption{Padding oracle attack on CBC mode}
    \label{fig:cbc-padding-oracle-attack}
\end{figure}

\paragraph{Predictable IVs}
Random IVs are not just theoretically relevant for IND-CPA security of CBC mode.
Consider the following steps (see \autoref{fig:cbc-predictable-iv}):
\begin{enumerate}
\item $\A$ queries LoR oracle with $(P_0, P_1)$
\item Oracle responds with $C = C_0||C_1$ where $C_1 = E_K(P_b \xor C_0)$
\item $\A$ predicts next $IV=C_0'$
\item $\A$ queries $P_0 \xor C_0 \xor C_0'$
\item $\implies b = 0 \iff P_b = P_0 \iff C_1 = C_1' \implies$ breaks IND-CPA security
\end{enumerate}

In practice: systems may use \emph{IV chaining} (use the last ciphertext as the next IV, in order to avoid having to sample new randomness).
E.g.: SSLv3, TLS 1.0, SSH in CBC mode.
See also the BEAST attack on TLS.

\begin{figure}[h]
    \centering
	\includegraphics[scale=0.6]{images/cbc-predictable-iv.png}
    \caption{Attack on predictable IVs in CBC mode}
    \label{fig:cbc-predictable-iv}
\end{figure}



\subsection{Hash Functions}

\paragraph{Cryptographic Hash Function}
An efficiently computable function $H: \setzeroone^* \mapsto \setzeroone^n$ that maps arbitrary-length inputs to fixed-length outputs (``\emph{message digests}'').
Not keyed!

Applications: MACs, signatures, key derivation, commitments.

\paragraph{Random oracle model}
Model under which hash functions are assumed to produced outputs that are uniformly random distributed.
I.e. a hash function could be modelled by a random oracle.
Very strong assumption!

\paragraph{Birthday paradox}
When drawing elements at random from a set of size $s$ then after $\sqrt{s}$ trials we expect a collision with 39\% probability.
We quickly get to 99\% with an additional constant factor.

\paragraph{Security goals}
Primary goals:
\begin{itemize}
	\item \emph{Pre-image resistance} (one-wayness):
	Given $h$, it is infeasible to find an $m$ such that $H(m) = h$.
	\item \emph{Second pre-image resistance}:
	Given $m_1$, it is infeasible to find an $m_2$ such that $H(m_1) = H(m_2)$.
	\item \emph{Collision resistance}:
	It is infeasible to find any $m_1 \neq m_2$ such that $H(m_1) = H(m_2)$.
\end{itemize}

Secondary goals:
\begin{itemize}
	\item \emph{Near-collision resistance}:
	It is infeasible to find any $m_1 \neq m_2$ such that $H(m_1) \approx H(m_2)$ (e.g. hashes agree on most bits).
	\item \emph{Partial pre-image resistance 1}:
	Given $H(m)$, it is infeasible to recover any partial information about $m$.
	\item \emph{Partial pre-image resistance 2}:
	Given a target string $t$ with $|t|=l$ it is infeasible to find an $m$ such $H(m)=t||x$ (faster than with $2^l$ brute-force hash evaluations).
\end{itemize}

CR adversary:\\
Cannot quantify security over \underline{all} efficient $\A$ (collisions exist, $\A$ can hardcode one).
Instead, define ``$(t, \varepsilon)$-CR adversaries'', running in time $t$ and with $\adv^{CR}_H(\A) = \varepsilon$.
Build reductions from there.

Relationships:
\begin{itemize}
\item Collision resistance $\implies$ second pre-image resistance
\item Maybe: Collision resistance $\implies$ pre-image resistance -- depending on how you define pre-image resistance (sampling from the domain or from the range?)
\end{itemize}

Generic attacks (in the ROM):\\
(Second) pre-image resistance: $2^n$ evaluations.\\
Collision resistance: $2^{n/2}$ evaluations! (by the birthday paradox)\\
$\implies$ e.g. SHA-1 with 160-bit outputs only achieves 80-bit security

\paragraph{Merkle-Damgård iterated hashing}
Constructs a hash function $H$ from a \emph{compression function} \mbox{$h: \setzeroone^k \times \setzeroone^n \mapsto \setzeroone^n$}.
Used e.g. in MD5, SHA-1, SHA-2.

Steps: pad message, split into k-bit chunks, repeatedly apply $h$ to the chunks and IV/chaining values (see \autoref{fig:merkle-damgard}).

Padding scheme:
$$ m' = pad(m) = m || 10^t || [len(m)]_k $$
where $[len(m)]_k$ is the k-bit encoding of the message length\footnote{This limits the maximum length of a message that can be hashed to $2^k - 1$.}
and $0 \leq t < k$ is minimal.

Theorem: If $h$ is collision resistant, then so is $H$.%
\footnote{Note that the choice of padding scheme matters for the proof!
Also note that the two IVs must be the same for a \emph{full collision}.
The much weaker form of two colliding messages for different IVs is called a \emph{freestart collision}.}

Length extension attack:
Given $H(m)$ (but not $m$!) once can compute a valid hash $y=H(pad(m)||m'')$.
This is problematic e.g. when construction MACs or KDFs from a hash function.

\begin{figure}[h]
    \centering
	\includegraphics[scale=0.4]{images/merkle-damgard.png}
    \caption{Merkle-Damgård transform}
    \label{fig:merkle-damgard}
\end{figure}

\paragraph{Constructing compression functions from block ciphers}
E.g. Davies-Meyer, Matyas-Meyer-Oseas, Miyaguchi-Preneel constructions.
Use message as the key (need fast rekeying!) and (some variation of) chaining value as ``plaintext'' input.

\begin{figure}[h]
    \centering
	\includegraphics[scale=0.35]{images/davies-meyer.png}
    \caption{Davies-Meyer construction}
    \label{fig:davies-meyer}
\end{figure}

\paragraph{Sponge construction}
Different design approach.
Centered around 2 phases: absorbing + squeezing.
Key ideas: giant bit permutation $F$, not inputting/outputting entire state.
Variable length output.

E.g. used in SHA-3/Keccak: $\text{SHA-3}(m)=out_1||out_2||out_3||...$.

\begin{figure}[h]
    \centering
	\includegraphics[scale=0.35]{images/sponge-absorb.png}
    \caption{Sponge construction: absorbing}
    \label{fig:sponge-absorb}
\end{figure}
\begin{figure}[h]
    \centering
	\includegraphics[scale=0.35]{images/sponge-squeeze.png}
    \caption{Sponge construction: squeezing}
    \label{fig:sponge-squeeze}
\end{figure}



\subsection{Message Authentication Codes MACs}


\subsection{Authenticated Encryption}

